\section{Tensores}
\subsection{Convenio de sumación de Einstein}
Antes de comenzar conviene explicar el \textbf{convenio de sumación de Einstein}, que nos va a permitir escribir expresiones de forma compacta que utilizando el símbolo de sumatorio, $\sum$, serían muy largas.

Tomemos por ejemplo un vector \lrg{$\bm{v}$} expresado en función de una base \lrg{$\bm{e}_i$}, con componentes \lrg{$v^i$}, donde \lrg{$i$} no es un exponente, sino un superíndice, esta elección es la clave del criterio.
\lrg{\[\bm{v}=\sum_{i=1}^n{v^i \bm{e}_i}=\sum{v^i \bm{e}_i}\equiv v^i \bm{e}_i\]}
Para poder aplicar el criterio tenemos que tener claro cuales son los posibles valores que puede tomar el índice, en este caso, desde 1 hasta la dimensión del espacio vectorial.

La clave está en que cuando tenemos un superíndice y un súbindice iguales en un mismo término, esto indica una suma con respecto a ese índice, denominado \textbf{índice mudo}, porque solo indica la suma.

Otro ejemplo es aplicado a formas bilineales o endomorfismos, los detalles se detallan posteriormente.
\lrg{\[\phi(\bm{v},\bm{w})=A_{ij} v^i w^j \;\;\;\;\; {f(\bm{v})}^j={M^j}_i \ v^i\]}
Se observa también que en una ecuación los índices no mudos a ambos lados deben coincidir.
\subsection{Covarianza y Contravarianza}
Las componentes de un vector de un espacio vectorial \lrg{$V$} se indican con un superíndice, mientras que los vectores de la base con un subíndice. Por el contrario, las componentes de un vector de \lrg{$V^*$}, el dual, se indican con un subíndice y la base dual se indica con superíndices.
\subsubsection{Base Dual}
\vspace{-25pt}
\lrg{\begin{equation}
\bm{e}^i (\bm{e}_j)= \delta^i_j
\end{equation}}
Recordemos que la base dual se define de esta forma, donde \lrg{$\bm{e}^i$} son los vectores de la base dual, \lrg{$\bm{e}_j$} son los vectores de una base cualquiera de \lrg{$V$}, y \lrg{$\delta^i_j$} es el delta de Kronecker, así pues si los indices son iguales el resultado será 1 y si son distintos será 0.
\subsubsection{Métrica}
De nuevo, recordemos que para medir distancias, recurrimos a un producto escalar, que define una métrica euclidea, de tal forma que
\lrg{\begin{equation}
|\bm{v}|=\sqrt{\phi(\bm{v},\bm{v})}=\sqrt{g_{ij} v^i v^j}
\end{equation}}
De tal forma que \lrg{$g$} es una matriz que define al producto escalar que llamamos la métrica o el tensor métrico, y \lrg{$g_{ij}$} son sus componentes, las cuales dependen de la base de los vectores, de la siguiente manera
\lrg{\begin{equation}
g_{ij} = \bm{e}_i \bm{\cdot} \bm{e}_j
\end{equation}}
\vspace{-35pt}
\subsubsection{Subir y bajar índices}
En el caso \lrg{$\R$}, en el que trabajamos, podemos definir un isomorfismo entre \lrg{$V$} y \lrg{$V^*$} tal que \lrg{$\bm{v}\mapsto \phi(\bm{v},-)$}, dónde \lrg{$\bm{v} \in V$} y \lrg{$\phi(\bm{v},-)=g_{ij} v^i \bm{e^j}$}, donde \lrg{$\bm{e^j}$} es la base dual, así llegamos a
\lrg{\begin{equation}
\begin{matrix}
v_j = g_{ki} v^i && v^j = g^{ji} v_i \\
\bm{e}^j = g^{ji} \bm{e}_i && \bm{e}_j = g_{ji} \bm{e}^i
\end{matrix} \;\;\;\;\; g^{ij} = \left(g_{ij}\right)^{-1} \iff g^{ij} g_{jk} = \delta^i_k
\label{1.4}
\end{equation}}
Esto es importante pues nos permite pasar las coordendas de un vector de \lrg{$V$} a \lrg{$V^*$} y viceversa, o interpretar los elementos de \lrg{$V^*$} como elementos de \lrg{$V$}.
\subsubsection{Covarianza y Contravarianza}
Los superíndices se denominan elementos contravariantes, mientras que los subíndices se denominan elementos covariantes. Esto indica como se transforman estos elementos cuando realizamos un cambio de base.

Si tomamos dos bases de \lrg{$V$}, tenemos que \lrg{$({v'})^j = {S^j}_i \ v^i$}, mientras que para un elemento de \lrg{$V^*$}, si suponemos que esta en una base original y lo pasamos a una base nueva, como lo podemos interpretar como una transformación lineal desde un elemento de \lrg{$V$} hasta \lrg{$\R$}, para que cambiar de base otra, tenemos que multiplicar por la matriz de cambio de base de la base nueva a la antigua, que es la inversa de \lrg{$S$}, pues \lrg{$S$} es la matriz de cambio de base de la antigua a la nueva, así, llegamos a que \lrg{$(v')_j = {(S^{-1})_j}^i \ v_i$}.

Así, los elementos covariantes y contravariantes, al realizar un cambio de base, se transforman de forma inversa.

En el ejemplo que he mostrado se ve como como las componentes de un vector de \lrg{$V$} son contravariantes y las de un vector de \lrg{$V^*$} son covariantes.

Otro ejemplo es por ejemplo las coordenadas de un vector de \lrg{$V$} en una base, que son contravariantes, y los vectores de una base expresados en otra, que son covariantes, puesto que si S es la matriz de cambio de la base vieja a la nueva, su inversa, la matriz de cambio de la base nueva a la vieja contiene como columnas a los vectores de la base nueva expresados en función de la base vieja, así tenemos \lrg{$(\bm{e}')_j = {(S^{-1})^i}_j \ \bm{e}_i$}.